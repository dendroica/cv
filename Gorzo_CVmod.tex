%!TEX TS-program = xelatex
%!TEX encoding = UTF-8 Unicode
% Awesome CV LaTeX Template for CV/Resume
%
% This template has been downloaded from:
% https://github.com/posquit0/Awesome-CV
%
% Author:
% Claud D. Park <posquit0.bj@gmail.com>
% http://www.posquit0.com
%
%
% Adapted to be an Rmarkdown template by Mitchell O'Hara-Wild
% 23 November 2018
%
% Template license:
% CC BY-SA 4.0 (https://creativecommons.org/licenses/by-sa/4.0/)
%
%-------------------------------------------------------------------------------
% CONFIGURATIONS
%-------------------------------------------------------------------------------
% A4 paper size by default, use 'letterpaper' for US letter
\documentclass[11pt,a4paper,]{awesome-cv}

% Configure page margins with geometry
\usepackage{geometry}
\geometry{left=1.4cm, top=.8cm, right=1.4cm, bottom=1.8cm, footskip=.5cm}


% Specify the location of the included fonts
\fontdir[fonts/]

% Color for highlights
% Awesome Colors: awesome-emerald, awesome-skyblue, awesome-red, awesome-pink, awesome-orange
%                 awesome-nephritis, awesome-concrete, awesome-darknight

\colorlet{awesome}{awesome-red}

% Colors for text
% Uncomment if you would like to specify your own color
% \definecolor{darktext}{HTML}{414141}
% \definecolor{text}{HTML}{333333}
% \definecolor{graytext}{HTML}{5D5D5D}
% \definecolor{lighttext}{HTML}{999999}

% Set false if you don't want to highlight section with awesome color
\setbool{acvSectionColorHighlight}{true}

% If you would like to change the social information separator from a pipe (|) to something else
\renewcommand{\acvHeaderSocialSep}{\quad\textbar\quad}

\def\endfirstpage{\newpage}

%-------------------------------------------------------------------------------
%	PERSONAL INFORMATION
%	Comment any of the lines below if they are not required
%-------------------------------------------------------------------------------
% Available options: circle|rectangle,edge/noedge,left/right

\name{Jessica}{Gorzo}

\position{Research Associate}
\address{Cellular Tracking Technologies, Erma, NJ}

\pronouns{she/her}
\homepage{avianecologist.com}
\github{dendroica}
\linkedin{gorzo}
\twitter{setophaga}

% \gitlab{gitlab-id}
% \stackoverflow{SO-id}{SO-name}
% \skype{skype-id}
% \reddit{reddit-id}


\usepackage{booktabs}

\providecommand{\tightlist}{%
	\setlength{\itemsep}{0pt}\setlength{\parskip}{0pt}}

%------------------------------------------------------------------------------



% Pandoc CSL macros
% definitions for citeproc citations
\NewDocumentCommand\citeproctext{}{}
\NewDocumentCommand\citeproc{mm}{%
  \begingroup\def\citeproctext{#2}\cite{#1}\endgroup}
\makeatletter
 % allow citations to break across lines
 \let\@cite@ofmt\@firstofone
 % avoid brackets around text for \cite:
 \def\@biblabel#1{}
 \def\@cite#1#2{{#1\if@tempswa , #2\fi}}
\makeatother
\newlength{\cslhangindent}
\setlength{\cslhangindent}{1.5em}
\newlength{\csllabelwidth}
\setlength{\csllabelwidth}{3em}
\newenvironment{CSLReferences}[2] % #1 hanging-indent, #2 entry-spacing
 {\begin{list}{}{%
  \setlength{\itemindent}{0pt}
  \setlength{\leftmargin}{0pt}
  \setlength{\parsep}{0pt}
  % turn on hanging indent if param 1 is 1
  \ifodd #1
   \setlength{\leftmargin}{\cslhangindent}
   \setlength{\itemindent}{-1\cslhangindent}
  \fi
  % set entry spacing
  \setlength{\itemsep}{#2\baselineskip}}}
 {\end{list}}

\usepackage{calc}
\newcommand{\CSLBlock}[1]{\hfill\break\parbox[t]{\linewidth}{\strut\ignorespaces#1\strut}}
\newcommand{\CSLLeftMargin}[1]{\parbox[t]{\csllabelwidth}{\strut#1\strut}}
\newcommand{\CSLRightInline}[1]{\parbox[t]{\linewidth - \csllabelwidth}{\strut#1\strut}}
\newcommand{\CSLIndent}[1]{\hspace{\cslhangindent}#1}

\begin{document}

% Print the header with above personal informations
% Give optional argument to change alignment(C: center, L: left, R: right)
\makecvheader

% Print the footer with 3 arguments(<left>, <center>, <right>)
% Leave any of these blank if they are not needed
% 2019-02-14 Chris Umphlett - add flexibility to the document name in footer, rather than have it be static Curriculum Vitae
\makecvfooter
  {January 2025}
    {Jessica Gorzo~~~·~~~Curriculum Vitae}
  {\thepage}


%-------------------------------------------------------------------------------
%	CV/RESUME CONTENT
%	Each section is imported separately, open each file in turn to modify content
%------------------------------------------------------------------------------



\section{Recent Experience}\label{recent-experience}

\begin{cventries}
    \cventry{Research associate}{Conservation Science Global, Inc.}{Remote}{Mar. 2024 - Present}{\begin{cvitems}
\item Data wrangling and ecological analysis
\item Working on a variety of projects to complete grant goals and produce deliverables
\end{cvitems}}
    \cventry{Data scientist}{Cellular Tracking Technologies LLC}{Rio Grande, NJ}{Jul. 2020 - Present}{\begin{cvitems}
\item Develop methodology to estimate locations from radio and Bluetooth signals
\item Manage data pipelines and processes that deliver and display data via web interfaces
\end{cvitems}}
    \cventry{Landscape Ecologist}{Natural Resources Research Institute}{Duluth, MN}{Aug. 2016 - Jun. 2020}{\begin{cvitems}
\item Researcher 5 (University of Minnesota - NRRI internal job title designation)
\end{cvitems}}
\end{cventries}

\subsection{Contractor}\label{contractor}

\begin{cventries}
    \cventry{Marsh Bird Survey Technician}{New Jersey Dept. of Environmental Protection}{Tuckahoe, NJ}{Summer 2021, 2024}{\begin{cvitems}
\item Worked as a contractor to survey sensitive and endangered noctural avian species via boat transects
\item Solely by-ear bird identification required for night surveys
\end{cvitems}}
    \cventry{Co-Founder}{On the Wing Scientific Consulting LLC}{Remote}{Jul. 2018 - Present}{\begin{cvitems}
\item Data science consulting projects including spatio-temporal modeling
\item Create tools for improving workflows
\end{cvitems}}
\end{cventries}

\section{Education}\label{education}

\begin{cventries}
    \cventry{B.S. Biological Sciences}{Virginia Tech}{Blacksburg, VA}{2004-08}{\begin{cvitems}
\item Minors: Physics, Astronomy
\end{cvitems}}
    \cventry{M.S. Wildlife \& Fisheries Biology}{Clemson University}{Clemson, SC}{2009-11}{\begin{cvitems}
\item Minor: Experimental Statistics
\end{cvitems}}
    \cventry{Ph.D. Wildlife Ecology}{University of Wisconsin}{Madison, WI}{2012-16}{\begin{cvitems}
\item Minor: distributed (required)
\end{cvitems}}
\end{cventries}

\section{Thesis \& Dissertation}\label{thesis-dissertation}

\phantomsection\label{refs-ca0136a0be3f421da152aa9cf565cc7e}
\begin{CSLReferences}{1}{0}
\bibitem[\citeproctext]{ref-gorzo2016avian}
Gorzo, J. M. (2016). \emph{Avian response to weather in the central US
grasslands} {[}PhD thesis{]}. The University of Wisconsin-Madison.

\bibitem[\citeproctext]{ref-gorzo2012avian}
Gorzo, J. M. (2012). \emph{Avian communities and landscape
characteristics of golf courses within the Beaufort County sea island
complex} {[}Master's thesis{]}. Clemson University.

\end{CSLReferences}

\section{Peer-Reviewed Publications}\label{peer-reviewed-publications}

\phantomsection\label{refs-e27578b494d3bed8953c36fb56189219}
\begin{CSLReferences}{1}{0}
\bibitem[\citeproctext]{ref-gorzo2016using}
Gorzo, J. M., Pidgeon, A. M., Thogmartin, W. E., Allstadt, A. J.,
Radeloff, V. C., Heglund, P. J., \& Vavrus, S. J. (2016). Using the
North American Breeding Bird Survey to assess broad-scale response of
the continent's most imperiled avian community, grassland birds, to
weather variability. \emph{The Condor: Ornithological Applications},
\emph{118}(3), 502--512.

\bibitem[\citeproctext]{ref-schilke2020modeling}
Schilke, P. R., Bartrons, M., Gorzo, J. M., Vander Zanden, M. J.,
Gratton, C., Howe, R. W., \& Pidgeon, A. M. (2020). Modeling a
cross-ecosystem subsidy: Forest songbird response to emergent aquatic
insects. \emph{Landscape Ecology}, \emph{35}, 1587--1604.

\bibitem[\citeproctext]{ref-mcintyre2019simulating}
McIntyre, N., Liu, G., Gorzo, J. M., Wright, C. K., Guntenspergen, G.
R., \& Schwartz, F. (2019). Simulating the effects of climate
variability on waterbodies and wetland-dependent birds in the Prairie
Pothole Region. \emph{Ecosphere}, \emph{10}(4), e02711.

\bibitem[\citeproctext]{ref-hamilton2018slow}
Hamilton, C. M., Bateman, B. L., Gorzo, J. M., Reid, B., Thogmartin, W.
E., Peery, M. Z., Heglund, P. J., Radeloff, V. C., \& Pidgeon, A. M.
(2018). Slow and steady wins the race? Future climate and land use
change leaves the imperiled Blanding's turtle (Emydoidea blandingii)
behind. \emph{Biological Conservation}, \emph{222}, 75--85.

\bibitem[\citeproctext]{ref-latimer2022rising}
Latimer, C. E., Graves, R., Pidgeon, A. M., Gorzo, J. M., Henschell, M.
M., Schilke, P. R., Hobi, M. L., Olah, A., Kennedy, C. M., Zuckerberg,
B., et al. (2022). Rising novelty and homogenization of breeding bird
communities in the US. \emph{bioRxiv}, 2022--2009.

\bibitem[\citeproctext]{ref-simonetti2008minor}
Simonetti, J., Chou, A., Dowd, J., Gorzo, J., Grimm, C., Harold, M.,
Hubbard, R., Kuether, T., Loyd, R., Nipper, B., et al. (2008). Minor
Planet Observations {[}841 Martin Observatory, Blacksburg{]}.
\emph{Minor Planet Circulars}, \emph{62871}, 9.

\end{CSLReferences}

\section{Conference Presentations}\label{conference-presentations}

\phantomsection\label{refs-d1267b26d664c19ef27cb37d646a1e84}
\begin{CSLReferences}{1}{0}
\bibitem[\citeproctext]{ref-aos}
Gorzo, J. M., \& Burcher, S. (2024). Automated radio telemetry: Study
design and data analysis. \emph{Meeting of the American Ornithological
Society}.

\bibitem[\citeproctext]{ref-bls8}
Burcher, S., Blackshire, S., Gorzo, J., Lanzone, M., La Puma, D., \&
Mizrahi, D. (2024a). Tracking wildlife movements with networks of fixed
radio receivers. \emph{8th International Bio-Logging Science Symposium}.

\bibitem[\citeproctext]{ref-aostalk}
Burcher, S., Blackshire, S., Gorzo, J., Lanzone, M., La Puma, D., \&
Mizrahi, D. (2024b). Tracking wildlife movements with networks of fixed
radio receivers. \emph{Meeting of the American Ornithological Society}.

\bibitem[\citeproctext]{ref-jess2018mallard}
Gorzo, J. M., \& Wright, C. K. (2018). Mallard Distribution and Wetland
Network Modularity in the Prairie Pothole Region (PPR). \emph{IALE-NA},
\emph{T02}, Terrestrial--aquatic ecosystem interactions.

\bibitem[\citeproctext]{ref-tag}
Gorzo, J. M. (2017). Waterfowl response to weather and land cover in the
prairie potholes. \emph{Proceedings of the Ninety-Eighth Annual
Meeting}, 922. \url{https://doi.org/10.1676/1559-4491-129.4.911}

\bibitem[\citeproctext]{ref-tag2}
Gorzo, J. M., Pidgeon, A., Allstadt, A., Thogmartin, W., Vavrus, S., \&
Heglund, P. (2016). Land cover and weather drivers of avian site
fidelity. \emph{Proceedings of the Society for Conservation Biology}.

\bibitem[\citeproctext]{ref-wright2018mallard}
Wright, C. K., \& Gorzo, J. M. (2018). For mallard ducks, the network is
the habitat. \emph{AGU Fall Meeting Abstracts}, \emph{2018}, H21L--1826.

\bibitem[\citeproctext]{ref-aou}
Gorzo, J. M., Pidgeon, A. M., Thogmartin, W., Allstadt, A., \& Radeloff,
V. (2015). Grassland bird response to weather in the badlands and
prairies. \emph{Meeting of the American Ornithologists' Union/Cooper
Society}.

\bibitem[\citeproctext]{ref-earth}
Gorzo, J. M. (2014). Avian population response to extreme weather
events. \emph{Nelson Earth Day Conference}.

\bibitem[\citeproctext]{ref-aoufl}
Gorzo, J. M., \& Jodice, P. G. R. (2011). The breeding bird community of
coastal golf courses in Beaufort County, South Carolina. \emph{Meeting
of the American Ornithologists' Union}.

\bibitem[\citeproctext]{ref-wos}
Bateman, B., Gorzo, J. M., Pidgeon, A., Radeloff, V., Akcakaya, R.,
Flather, C., Albright, T., Vavrus, S., Thogmartin, W., \& Heglund, P.
(2013). Use of multiple data sources in identifying drivers of abundance
in an irruptive species, the dickcissel. \emph{Meeting of the American
Ornithologists' Union/Cooper Ornithological Society}.

\bibitem[\citeproctext]{ref-allstadt2016effects}
Allstadt, A. J., Gorzo, J. M., Bateman, B. L., Heglund, P. J., Pidgeon,
A. M., Thogmartin, W., Vavrus, S. J., \& Radeloff, V. (2016). Effects of
weather on the abundance and distribution on populations of 103 breeding
bird species across the United States. \emph{AGU Fall Meeting
Abstracts}, \emph{2016}, GC11D--1181.

\end{CSLReferences}

\section{Technical Reports}\label{technical-reports}

\phantomsection\label{refs-ca99a927caf9c71eb24e8f38ad257e5a}
\begin{CSLReferences}{1}{0}
\bibitem[\citeproctext]{ref-noe2021measuring}
Noe, R., Locke, C., Host, G., Gorzo, J., Johnson, L., Lonsdorf, E.,
Grinde, A., Joyce, M., Bednar, J., Dumke, J., et al. (2021).
\emph{Measuring what matters: Assessing the full suite of benefits of
OHF investments}.

\bibitem[\citeproctext]{ref-CitekeyTechreport}
Johnson-Bice, S., Gorzo, J. M., Kovalenko, K., Brown, T., \& Host, G.
(2020). \emph{Predicting Potential Beaver Dam Sites on Lake Superior's
North Shore}. Natural Resources Research Institute.

\end{CSLReferences}

\section{Contributions}\label{contributions}

\begin{itemize}
\item
  StellaR: \url{http://www.r-gis.net/stellar/}
\item
  Updated code for Supplement 1. R Script in 2016 for the publication:
  Jackson, M. M., Turner, M. G., Pearson, S. M., \& Ives, A. R. (2012).
  Seeing the forest and the trees: multilevel models reveal both species
  and community patterns. Ecosphere, 3(9), 1-16.
  \url{https://figshare.com/articles/dataset/Supplement_1_R_script_and_input_data_file_for_creating_multilevel_models_using_the_forest_herb_data_set_described_in_the_main_text_/3563784?file=5637069}
\item
  Contributed ``Raptors and songbirds'' (8.4.5.1) pp.565-566 in Jodice,
  P. G. R., Tavano, J., Mackin, W., \& Michel, J. (2013). South Atlantic
  information resources: data search and literature synthesis. US
  Department of the Interior, Bureau of Ocean Energy Management,
  regulation, and enforcement, Gulf of Mexico OCS region, 475, 587
\end{itemize}

\section{Teaching}\label{teaching}

\subsection{Guest Lectures}\label{guest-lectures}

\begin{cventries}
    \cventry{Career Perspective: From Wildlife to Data Science}{DSSA 5102: Data Gathering \& Warehousing}{Stockton University}{4/18/2024}{}\vspace{-4.0mm}
    \cventry{Networks and Connectivity in Landscape Ecology}{Landscape Ecology}{University of Minnesota-Duluth}{2020}{}\vspace{-4.0mm}
    \cventry{Introduction to Birding and Community Science}{WLDL 450/650: Human Dimensions in Wildlife}{University of Wisconsin-Stevens Point}{2016}{}\vspace{-4.0mm}
\end{cventries}

\subsection{Teaching Assistant}\label{teaching-assistant}

\begin{cvhonors}
    \cvhonor{}{Forestry Practicum ('Summer Camp')}{ Kemp Natural Resource Station}{2012, 2014}
    \cvhonor{}{Introduction to Astronomy Lab (2 Sections)}{Virginia Tech}{2008}
\end{cvhonors}

\subsection{Short Course \& Workshop
Instructor}\label{short-course-workshop-instructor}

\begin{cvhonors}
    \cvhonor{}{BIOSC 730: Wolves and Society}{Clemson University}{2013}
    \cvhonor{}{Introduction to Astronomy}{Elderhostel}{Fall 2007}
\end{cvhonors}

\section{Outreach, Service \&
Fundraising}\label{outreach-service-fundraising}

\begin{itemize}
\tightlist
\item
  Birding field guide (paid or volunteer) for various festivals

  \begin{itemize}
  \tightlist
  \item
    Cape May Fall Festival, Cape May Bird Observatory/NJ Audubon
    (2023-2024)
  \item
    Cape May Spring Festival, Cape May Bird Observatory/NJ Audubon
    (2024)
  \item
    Sax-Zim Bog Birding Festival (2014 - 2019)
  \end{itemize}
\item
  World Series of Birding: Participant in Carbon Footprint Challenge

  \begin{itemize}
  \tightlist
  \item
    2015: tied for 1st
  \item
    2021: 2nd place in social/physical distance COVID-19 modified rules
  \end{itemize}
\item
  Great Wisconsin Birdathon Participant (2013, 2016)
\item
  BRRRRDathon (Team Overall Winner, Record Setter Jan.~2013)
\end{itemize}

\subsection{Television}\label{television}

Expeditions -- air date Oct 17 2010\\
Guest appearance, season 4, Episode 7: ``The ACE Basin -- A Conservation
Miracle''\\
Discussed painted bunting habitat requirements, threats and conservation
needs\\
(Listing and TV show description:\\
\url{http://www.clemson.edu/public/expeditions/episode_guide/season_4.html}
)

\subsection{Radio}\label{radio}

Your Day -- air date Aug 30 2010\\
(Segment start 45:50
\url{http://cufan.clemson.edu/psaradiopod/YDPodcast/YD100830.mp3} )

\subsection{Academic}\label{academic}

\textbf{Ad hoc reviewer}: Journal of Arid Environments, Landscape \&
Urban Planning, Landscape Ecology, PLOS ONE, Wilson Ornithological
Society, Biological Conservation, Avian Conservation \& Ecology\\
\textbf{Professional Memberships}: Phi Kappa Phi Honors Society
(inducted 2010), American Ornithological Society, US-International
Association for Landscape Ecology

\section{Awards \& Honors}\label{awards-honors}

2015 Travel Award, American Ornithologist's Union/Cooper Ornithological
Society Conference\\
2011 Outstanding Service to the Clemson Experimental Forest\\
Dean's List (undergraduate)



\end{document}
